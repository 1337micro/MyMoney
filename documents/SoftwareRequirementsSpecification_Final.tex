\documentclass[12pt]{article}
\usepackage{graphicx}
\usepackage{array}
\usepackage{float}
\pagestyle{empty}
\setcounter{secnumdepth}{2}

\topmargin=0cm
\oddsidemargin=0cm
\textheight=22.0cm
\textwidth=16cm
\parindent=0cm
\parskip=0.15cm
\topskip=0truecm
\raggedbottom
\abovedisplayskip=3mm
\belowdisplayskip=3mm
\abovedisplayshortskip=0mm
\belowdisplayshortskip=2mm
\normalbaselineskip=12pt
\normalbaselines

\begin{document}
\pagestyle{plain}
\vspace*{0.5in}
\centerline{\bf\Large "My Money" Software Requirements Specification}

\vspace*{0.5in}
\centerline{\bf\Large Team PB-PK}

\vspace*{0.5in}
\centerline{\bf\Large 11 February 2018}

\vspace*{1.0in}
\begin{table}[ht]


\begin{center}
\begin{tabular}{|l | l |}
\hline
\textbf{Member Name} & \textbf{Member ID} \\

\hline Noemi Lemonnier & 40001085 \\ \hline Genevieve Plante-Brisebois & 40003112 \\ \hline Han Gao & 40053734 \\
\hline Theo	Grimond & 27276044 \\ \hline Real	Nguyen & 27566263 \\ \hline Ornela Bregu & 26898580 \\
\hline William	Prioriello & 27080956 \\ \hline Tiantian	Ji & 27781083 \\ \hline Dong-Son Nguyen-Huu & 40014054  \\
\hline Ashesh Patel & 40018519 \\ \hline Sabrina	Rieck & 40032864 \\

\hline
\end{tabular}
\caption{Team PB-PK}
\end{center}
\end{table}

\clearpage

\Large{\bf Revision History}
\vspace*{0.2in}
\begin{table}[ht]


\begin{center}
\begin{tabular}{| m{2cm} | m{2cm}| m{8cm} | m{3cm}|}
\hline
\textbf{Date} & \textbf{Version} & \textbf{Description} & \textbf{Author} \\

\hline 21/01/2018 & 1.0 & Document Start & Ornela Bregu\\ \hline 28/01/2018 & 1.1 & Added comments describing what needs to be done for each section. & Ornela Bregu  \\

\hline 30/01/2018 & 1.2 & Added the document purpose, business goals and Use Case Diagram. & Ornela Bregu  \\ \hline 04/02/2018 & 1.3 & Added the remaining use cases. Fixed formatting. Added figure descriptions for the domain model and the system overview & Dong-Son Nguyen-Huu \\ 

\hline 07/02/2018 & 1.4 & Edited introduction, use goals and added Functional Requirements. & Ornela Bregu \\ 

\hline 08/02/2018 & 1.5 & Added domain model diagram. & Ornela Bregu  \\ 

\hline 08/02/2018 & 1.6 & Added Data Dictionary and References. & Ashesh Patel \\

\hline 09/02/2018 & 1.7 & Added Non-Functional Requirements and fixed formatting. & Ornela Bregu  \\ 

\hline 11/02/2018 & 1.8 & Final quality checking for content and formatting. & Real Nguyen \\

\hline
\end{tabular}
\caption{Revision History}
\end{center}
\end{table}

\clearpage
\listoffigures
\listoftables
\clearpage

\tableofcontents
\clearpage

\section{Introduction} \\
 The purpose of this document is to define the user requirements for the desktop application "My Money" and to ensure that we deliver what the user really needs by mitigating the risks as much as possible. We start by specifying the user goals and the reasons why we are developing this application. Then, we continue explaining the domain concepts, their meaning in the context of our application and an overview of the program we are planning to build. In the last part, we define functional and non-functional requirements for this project.    \\ The intended audience of this document is described in the table below.
\begin{table}[ht]

\begin{center}

\begin{tabular}{| m{5cm} | m{10cm}|}

\hline
\textbf{Group of readers} & \textbf{Reasons for reading} \\
 \hline Users & To give feedback about the requirements \\
\hline System developers & To understand what functions and properties the system must contain \\
\hline Testers & To test the system against the requirements \\
\hline Project organization team & To follow-up the status of the project against the requirements \\
\hline Project coordinator & To follow-up the status of the project \\

\hline
\end{tabular}
\caption{Intended Audience}
\end{center}
\end{table}

\section{User Goals}

This section describes the reasons why we are developing the "My Money" application.

The application is a tool to help any user manage their finances, keep track of their cash spending, and discover where they are spending the most money to then make necessary adjustments.
This desktop application allows the user to quickly refer to their accounts, modify their budget categories, and input new transactions.

\section{System Overview}
\begin{figure}[h!]
\includegraphics[width=15cm, height=15cm]{image/UseCaseDiagram.jpg} \\

 \caption{Use Case Diagram}
 
 
 \end{figure}
 The use of "My Money" revolves around the user, who can use the application to update their cash spending by listing the transactions that they have entered and specify which categories they fall into; manage their available cards, see the type and balance of each card, and add or remove cards; or set a budget, which can either automatically allocate resources among various categories or allow the user to set them manually. \\


\section{Domain Concepts}
\begin{figure}[ht!]
\includegraphics[width=15cm, height=15cm]{image/DomainModel.png}\\
 
 \caption{Domain Model Diagram}
 
 
 \end{figure}
The use of "My Money" starts with the user authenticating their identity by entering a username and a password. If the information entered is correct, the user can access the application. From there, the user can either track their income and expenses incurred through transactions, or calculate their budget according to their available income. The transactions are then associated with their respective cards. Cards may be either credit or debit, and have a balance and a card number. Transactions are also used for cash spending to be used with the budget set by the user. Alternatively, the budget can be paid for through the use of the user's available cards.\\

\clearpage

\section{Functional Requirements}
The Functional Requirements section details the capabilities and functions that the "My Money" application must be capable of performing. These requirements will assure that the application will correctly and reliably perform its intended functions. This section will provide general, as well as specific requirements to be used in the design, testing, and validation of the application. The focus is on what the program must do. Details on how the components will be developed and how the program will operate will be defined later in the Software Design Specification.
Functional Requirements in this document are organized by use case. There are three basic use cases:
\begin{enumerate}
  \item Manage Cards
  \item Update Cash Spending
  \item Budgeting
\end{enumerate}
As our project follows an Agile approach to Unified Process, we are planning to improve the basic use cases with other important features on other iterations.
The following subsection will describe all the basic use cases.

\subsection{Basic Use Cases}

This subsection describes the three main use cases of the application. They are described using the casual degree of formality.


\subsubsection{Manage Cards} \label{uc:1}
\begin{center}

\begin{tabular}{| p{5cm} | p{10cm} |}

\hline



Use Case ID  & CardManagementA\\

\hline

Use Case Name  & Manage Cards\\

\hline

Summary  & Allow a user to view the cards associated with their profile, including viewing the cards' type and balance. The user can add cards and specify the card type. \\

\hline

Actors  & Any private individual\\

\hline

Pre-Condition(s) &  User has installed "My Money" on their system \\

\hline

Main Flow & User wants to view, remove, or add cards:\newline

1. User wants to view active cards associated with their account \newline
2. User launches application \newline
3. User views the details of the cards attached to their account, including their type and balance \newline
4. User can add or remove cards\\

\hline

Alternate Flow(s) & 1. User adds a card and choose its type and starting balance \newline
2. User removes an existing card from their profile \\
\hline
Exception Flow(s) & User inputs invalid value (e.g. non-numeric or negative value) as the starting balance of an added card:\newline
1. Application will prompt user to enter a non-negative numeric value\\
\hline



\end{tabular}

\end{center}





\subsubsection{Cash Spending} \label{uc:2}

\begin{center}
\begin{tabular}{| p{5cm} | p{10cm} |}
\hline

Use Case ID  & CashSpendingA\\
\hline
Use Case Name  & Update Cash Spending\\
\hline
Goal/Purpose  & Allow a user to manage their cash spending on different categories\\
\hline
Actors  & Any private individual \\
\hline
Pre-Conditions &  User has installed "My Money" desktop application \\
\hline
Post-Conditions & Cash expenses are updated  \\
\hline
Main Flow & 
1. User wants to update their expenses\newline 
2. User starts application\newline
3. User enters amount they spent on any selected category\newline
4. Application stores value and in the selected category of expenses
\\
\hline
Alternate Flow(s) & 
At any time, Application fails: \newline
   - User restarts Application.
\\
\hline
Exception Flow(s) & User inputs invalid value (e.g. non-numeric or negative value) as the expense: \newline
1. Application will prompt user to enter a non-negative numeric value\\
\hline

\end{tabular}
\end{center}

\subsubsection{Budgeting} \label{uc:3}
\begin{center}
\begin{tabular}{| p{5cm} | p{10cm} |}
\hline

Use Case ID  & BudgetManagementA\\
\hline
Use Case Name  & Budgeting\\
\hline
Goal/Purpose  & Allow a user to manage their budget\\
\hline
Actors  & Any private individual\\
\hline
Pre-Conditions &  User has installed "My Money" on their system \\
\hline
Post-Conditions & Available amount is calculated and displayed for each category \\
\hline
Main Flow & User wants details on the usage of their funds:\newline
1.  User opens the "My Money" application and presses on "Budgeting" to calculate their budget \newline
2.  The user enters their available funds \newline
3. The application returns the amount for each category, calculated according to the default recommended percentages and the input\newline
\\
\hline
Alternate Flow(s) & 
At any time, Application fails: \newline
   - User restarts Application.
\\
\hline
Exception Flow(s) & User inputs invalid value (e.g. non-numeric or negative value) as the expense: \newline
1. Application will prompt user to enter a non-negative numeric value\\
\hline

\end{tabular}
\end{center}

\subsection{Other Use Cases}

For the following iterations, we are planning to make the whole application more user friendly. The user interface of the application will be more attractive, how the data is presented will be more interesting by giving the user more options, etc.
Listed below are added functionalities that will be added in future iterations.
Please note that as we are following an Agile methodology, the use cases below are subject to change or may require more information before adding details. 

\subsubsection{Update Card-Payments}
\begin{enumerate}
  \item User wants to link their expenses with the card they used for the payment.
  \item User opens the application and goes to the cash spending feature. 
  \item User enters the amount spent and selects the card they used for the payment.
  \item Application validates and records the amount to a user-specified card and updates the total amount spent for that card.
\end{enumerate}

\subsubsection{Control Expenses}
\begin{enumerate}
  \item User wants to link their expenses with the budget and get notified if they go over budget.
  \item User opens the application and goes to the cash spending feature. 
  \item User enters the amount spent and other related information.
  \item System validates and records the amount to specified category, checks the planned budget and generates a notification with amount left to spend or a budget warning.
\end{enumerate}

\subsubsection{Control Budgeting}
\begin{enumerate}
  \item The user wants to change the percentages of each category in their budget. 
  \item User opens the application and goes to the budgeting feature. 
  \item User changes the percentage of each category.
  \item System validates and records each percentage to each specified category. 
  \item System updates the budget.
\end{enumerate}

\subsubsection{Authenticate User}
User wants to authenticate themselves with a username and a password.

\subsubsection{Manage Budget}
User wants to add/delete categories and put a fixed amount for any of them. 

\subsubsection{Real-time Transactions}
User wants to link the application with real-time transactions.


\section{Non-Functional Requirements}
In this section, we describe the quality attributes and characteristics of the software. Most of the quality attributes come from the FURPS+ model.

\begin{table}[H]
\begin{center}
\begin{tabular} {|p{1cm}|p{1cm}|p{7cm}| p{2cm}|p{3cm}|}
\hline
\textbf{ID} & \textbf{Vers} & \textbf{Description} & \textbf{Priority}  & \textbf{Traces to use cases} \\
\hline 1 & 1 & Only registered users are allowed to use the application. & Must-have & Authenticate Users \\

\hline 2 & 1 & The source code shall be self-documented by placing the design description in a Javadoc-readable method header.  & Must-have & All use cases \\

\hline 3 & 1 & If the input data format changes, the developer will be able to make the required changes in less then 24 person-hours.  & Must-have & All use cases \\

\hline 4 & 1 & There can be no unhandled exceptions from incorrect user input.  & Must-have & All use cases \\

\hline 5 & 1 & All menus must have a consistent format.  & Must-have & All use cases \\

\hline 6 & 1 & A user can install and operate the program without assistance of any kind.  & Must-have & All use cases \\

\hline 7 & 1 & During a system restart, the system will return to a functioning state.   & Must-have & All use cases \\

\hline 8 & 1 & 90\% of novice users can learn to operate major use cases without outside assistance.   & Must-have & All use cases \\

\hline 9 & 1 & The main use cases must be accessible from the top screen.  & Must-have & All use cases \\

\hline 10 & 1 & The program should be available 24/7.   & Must-have & All use cases \\

\hline 11 & 1 & The finished software must support new users without needing to be rewritten or recompiled. & Must-have & All use cases \\



\hline
\end{tabular}
\caption{Non-Functional Requirements}
\end{center}
\end{table}

\clearpage


\section{Data Dictionary}

\begin{table}[H]
\begin{center}
\begin{tabular} {|p{3.5cm}|p{11.9cm}|}

\hline \textbf{Term} & \textbf{Definition} \\
\hline User & A person who uses the program for their own personal use \\
\hline The System  & The application\\
\hline Transactions & Cash flow (income and expense)\\
\hline Cards & Virtual currency object linked to bank account\\
\hline Cash Spending & Expenditure in cash\\
\hline Category & Predefined budget options \\
\hline FURPS+ & Model for classifying software quality attributes\\
\hline Domain Model & Depiction of relationship between entities (person, concept, object or event) \\
\hline Use Case Diagram & Depiction of association between actor and use cases \\
\hline
\end{tabular}
\caption{Data Dictionary}
\end{center}
\end{table}

\section{References}
Leave Debt Behind. (2010). 10 Recommended Category Percentages for Your Family Budget. [online] Available at: http://www.leavedebtbehind.com/frugal-living/budgeting/10-recommended-category-percentages-for-your-family-budget/ [Accessed 9 Feb. 2018].

\appendix


\end{document}
