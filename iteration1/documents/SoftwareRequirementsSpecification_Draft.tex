\documentclass[12pt]{article}
\usepackage{graphicx}
\usepackage{array}
\usepackage{float}
\pagestyle{empty}
\setcounter{secnumdepth}{2}

\topmargin=0cm
\oddsidemargin=0cm
\textheight=22.0cm
\textwidth=16cm
\parindent=0cm
\parskip=0.15cm
\topskip=0truecm
\raggedbottom
\abovedisplayskip=3mm
\belowdisplayskip=3mm
\abovedisplayshortskip=0mm
\belowdisplayshortskip=2mm
\normalbaselineskip=12pt
\normalbaselines

\begin{document}
\pagestyle{plain}
\vspace*{0.5in}
\centerline{\bf\Large Requirements Document}

\vspace*{0.5in}
\centerline{\bf\Large Team PB-PK}

\vspace*{0.5in}
\centerline{\bf\Large 11 February 2018}

\vspace*{1.5in}
\begin{table}[ht]


\begin{center}
\begin{tabular}{|l | l |}
\hline
\textbf{Name} & \textbf{ID Number} \\

\hline Noemi Lemonnier & 40001085 \\ \hline Genevieve Plante-Brisebois & 40003112 \\ \hline Han Gao & 40053734 \\
\hline Theo	Grimond & 27276044 \\ \hline Real	Nguyen & 27566263 \\ \hline Ornela Bregu & 26898580 \\
\hline William	Prioriello & 27080956 \\ \hline Tiantian	Ji & 27781083 \\ \hline Dong-Son Nguyen-Huu & 40014054  \\
\hline Ashesh Patel & 40018519 \\ \hline Sabrina	Rieck & 40032864 \\

\hline
\end{tabular}
\end{center}
\caption{Team PB-PK}
\end{table}

\clearpage

\begin{table}[ht]


\begin{center}
\begin{tabular}{| m{2cm} | m{2cm}| m{8cm} | m{3cm}|}
\hline
\textbf{Date} & \textbf{Version} & \textbf{Description} & \textbf{Author} \\

\hline 21/01/2018 & 1.0 & Document Start & Ornela Bregu\\ \hline 28/01/2018 & 1.1 & Added comments describing what needs to be done for each section. & Ornela Bregu  \\
\hline 30/01/2018 & 1.2 & Added the document purpose, business goals and Use Case Diagram. & Ornela Bregu  \\ \hline 04/02/2018 & 1.3 & Added the remaining use cases. Fixed formatting. Added figure descriptions for the domain model and the system overview & Dong-Son Nguyen-Huu \\ 
\hline 07/02/2018 & 1.4 & Edited introduction,  use goals and added Functional Requirements. & Ornela Bregu \\ 
\hline 08/02/2018 & 1.5 & Added domain model diagram & Ornela Bregu  \\ 
\hline 08/02/2018 & 1.6 & Added Data Dictionary and References & Ashesh Patel \\

\hline 09/02/2018 & 1.7 & Added Non-Functional Requirements and fixed formatting & Ornela Bregu  \\ 


\hline
\end{tabular}
\end{center}
\caption{Revision History}
\end{table}

\clearpage
\listoffigures

\listoftables
\clearpage

\tableofcontents



\clearpage

\section{Introduction} \\
 The purpose of this document is to define the user requirements for the desktop application "My Money" and to ensure that we deliver what user really needs by mitigating as much as possible the risks. We start by specifying the user goals and the reasons why we are developing this application. Then, we continue explaining the domain concepts, their meaning on our application context and an overview of the program we are planning to build. In the last part, we define functional and non-functional requirements for this project.    \\ The intended audience of this document is described in table 3. 
\begin{table}[ht]

\begin{center}

\begin{tabular}{| m{5cm} | m{10cm}|}

\hline
\textbf{Group of the readers} & \textbf{Reasons for reading} \\
 \hline Users & To give feedback about the requirements \\
\hline System developers & To understand what functions and properties the system must contain \\
\hline Testers & To test the system against the requirements \\
\hline Project organization team & To follow-up the status of the project against the requirements \\
\hline Project Coordinator & To follow-up the status of the project \\

\hline
\end{tabular}
\end{center}
\caption{Intended audience of this document} 
\end{table}

\section{User Goals}

This section describes the reasons why we are developing the "My Money" application.

The application is a tool to help any user manage his/her finances, keep track of his/her spending and discover where he/she is spending more money and make necessary changes so that he/she stop overspending.
This desktop app allows the user to quickly refer to his/her accounts, modify his/her budget categories and enter new transactions.

\section{System Overview}
\begin{figure}[h!]
\includegraphics[width=15cm, height=15cm]{image/UseCaseDiagram.jpg} \\

 \caption{User Case Diagram}
 
 
 \end{figure}
 The use of "MyMoney" revolves around the user, who can use the application to Update his cash spending (by listing the transactions that they have done and specify which categories they fall into), manage his available cards (and see what type they are, along with the amount available, and add/remove cards too), or set out a budget (which can either automatically allocate resources among various categories or allow the user to set them manually themselves). \\

\section{Domain Concepts}
 \begin{figure}[h!]
\includegraphics[width=15cm, height=15cm]{image/DomainModel.png}\\
 
\caption{Domain Model Diagram}
 
 
 \end{figure}
    The use of "MyMoney" starts with the user authenticating their identity by entering a username and password. If the information entered is correct, the user can access the app. From there, the user can either track their income and expenses incurred through transactions, or calculate their Budget according to their available income. The transactions are then associated with with their Cards, which may be of different type (credit and debit), and each card has the amount available and a card number. The resulting Transactions is also used for Cash Spending to be used according to the Budget that was previously calculated. Alternatively, the Budget can be paid for through the use of the user's available Cards.\\

\section{Functional Requirements}
The Functional Requirements section details the capabilities and functions that the "My Money" application must be capable of performing. These requirements will assure that the application will correctly and reliably perform its intended functionality. This section will provide general, as well as specific requirements to be used in the design, testing and validation of the application. The focus is on what the program must do; details on how the components will be developed and how it will operate will be defined later in the Software Design Specification.
Functional requirements in this document are organized by use cases. There are three basic use cases:
\begin{enumerate}
  \item Update Cash Spending
  \item Manage Cards
  \item Budgeting
\end{enumerate}
As our project follows an agile approach to Unified Process, we are planning to improve the basic use cases with other important features on other iterations.
In the following subsection are described all use cases.

\subsection{Basic Use Cases}

As previously described, in this subsection are described the three main use cases of our application. They are described using the casual degree of formality.


\subsubsection{ Manage Cards } \label{uc:1}
\begin{center}

\begin{tabular}{| p{5cm} | p{10cm} |}

\hline



Use Case ID  & CardManagementA\\

\hline

Use Case Name  & Manage Cards\\

\hline

Summary  & Allows a user to view the cards associated with their profile, including seeing each card's type and balance. The user can add cards and specify the type. \\

\hline

Actors  & Any private individual\\

\hline

Pre-Conditions &  User has installed "My Money" on their system \\


\hline

Main Flow & User wants to view, remove or add cards\newline

1.	User wants to view active cards associated with their account: \newline
2.  User launches  application \newline
3.  User views the cards attached to their account, including their type and balance \newline
4.	User can add or remove cards\\

\hline

Alternate Flow(s) & 1. User adds a card and choose its type and starting balance \newline
2. User removes an existing card from their profile \\
\hline
Exception Flow(s) & User enters an invalid input (ex: string or negative value) as the starting balance of an added card:\newline
- Application will prompt user to re-enter another value.\\
\hline



\end{tabular}

\end{center}





\subsubsection{Cash Spending} \label{uc:2}

\begin{center}
\begin{tabular}{| p{5cm} | p{10cm} |}
\hline

Use Case ID  & CashSpendingA\\
\hline
Use Case Name  & Update Cash Spending\\
\hline
Goal/Purpose  & Allow a user to manage his/her cash spending on different categories.\\
\hline
Actors  & Any private individual \\
\hline
Pre-Conditions &  User has installed "My Money" desktop application. \\
\hline
Post-Conditions & Cash expenses are updated.  \\
\hline
Main Flow & 
1. User wants to update his/her expenses.\newline 
2. User starts application.\newline
3. User enters amount he spent on any selected category.\newline
4. Application stores value in expenditure's table. 
\\
\hline
Alternate Flow(s) & 
At any time, Application fails: \newline
   - User restarts Application.
\\
\hline
Exception Flow(s) & User enters an invalid input (ex: string or negative value) \newline
- Application will prompt user to re-enter another value.\\
\hline

\end{tabular}
\end{center}

\subsubsection{Budgeting} \label{uc:3}
\begin{center}
\begin{tabular}{| p{5cm} | p{10cm} |}
\hline

Use Case ID  & BudgetManagementA\\
\hline
Use Case Name  & Budgeting\\
\hline
Goal/Purpose  & Allow a user to manage their budget at their own discretion or recommendation\\
\hline
Actors  & Any private individual\\
\hline
Pre-Conditions &  User has installed "My Money" on their system \\
\hline
Post-Conditions & Available amount is calculated and displayed for each category. \\
\hline
Main Flow & User wants a breakdown for using their funds\newline
1.  User opens the MyMoney application and presses on Budgeting to calculate their budget. \newline
2.  The user enters his/her available funds. \newline
3. The application returns the amount for each category, calculated according to the default recommended percentages and the input. (\newline
\\
\hline
Alternate Flow(s) & 
At any time, Application fails: \newline
   - User restarts Application.
\\
\hline
Exception Flow(s) & User enters an invalid input ( string,special character or a negative number) as his/her available funds. \newline
- Application will prompt user to re-enter another value.\\
\hline

\end{tabular}
\end{center}

\subsection{Other Use Cases}

For iteration 2, we are planning to make the whole application more user friendly. The user interface of the application should more attractive, how data is presented should be more interesting by giving the user more options, etc.
However, the added functionality during this iteration are listed below.

\subsubsection{Update Card-Payments}
User wants to link his/her expenses with the card he used for the payment.

User opens the application and goes to the Cash Spending feature. 
User enters the amount spent and selects the card he used for the payment.
System validates and records the amount to specified card and updates the total amount spent for that card.

\subsubsection{Control Expenses}
User wants to link his expenses with the budget and get notified if does not respect the budget.

User opens the application and goes to the Cash Spending feature. 
User enters the amount spent and other related information.
System validates and records the amount to specified category, checks the planned budget and generates a notification with amount left to spend or a budget warning.

\subsubsection{Control Budgetting}
The user wants to change the percentages of each category in his budget. 

User opens the application and goes to the Budgetting feature. 
User changes the percentage of each category.
System validates and records each percentage to each specified category.  
System updates the budget.

\subsubsection{Authenticate User}
User wants to authenticate himself with a username and password.

\subsubsection{Manage Budget}
User wants to add/delete categories and to put a fixed amount for any of them. 

\subsubsection{Real-time Transactions}
User wants to link the application with real-time transactions.

\section{Non-Functional Requirements}
In this section, we describe the quality attributes and characteristics of the software. Most of the quality attributes come from the FURPS+ model.

\begin{table}[H]
\begin{center}
\begin{tabular} {|p{1cm}|p{1cm}|p{7cm}| p{2cm}|p{3cm}|}
\hline
\textbf{ID} & \textbf{Vers} & \textbf{Description} & \textbf{Priority}  & \textbf{Traces to use cases} \\
\hline 1 & 1 & Only registered users are allowed to use the application. & Must & Authenticate Users \\

\hline 2 & 1 & The source code shall be self-documented by placing the design description in a Javadoc-readable method header.  & Must & All use cases \\

\hline 3 & 1 & If the input data format changes, the developer will be able to make the required changes in less then 24 person-hours.  & Must & All use cases \\

\hline 4 & 1 & There can be no unhandled exceptions from incorrect user input.  & Must & All use cases \\

\hline 5 & 1 & All menus must have a consistent format.  & Must & All use cases \\

\hline 6 & 1 & A user can install and operate the program without assistance of any kind.  & Must & All use cases \\

\hline 7 & 1 & During a system restart, the system will return to a functioning state.   & Must & All use cases \\

\hline 8 & 1 & 90\% of novice users can learn to operate major use cases without outside assistance.   & Must & All use cases \\

\hline 9 & 1 & The main use cases must be accessible from the top screen.  & Must & All use cases \\

\hline 10 & 1 & The program should be available 24/7.   & Must & All use cases \\

\hline 11 & 1 & The finished software must support new users without needing to be rewritten or recompiled. & Must & All use cases \\



\hline
\end{tabular}
\caption{Non-Functional Requirements}
\end{center}
\end{table}

\clearpage


\section{Data Dictionary}

\begin{table}[H]
\begin{center}
\begin{tabular} {|p{3.5cm}|p{11.9cm}|}

\hline \textbf{Term} & \textbf{Definition} \\
\hline User & A person who uses the program for his own personal use \\
\hline The System  & The application\\
\hline Transactions & Cash flow (income and expense)\\
\hline Cards & Virtual currency object linked to bank account\\
\hline Cash Spending & Expenditure in cash\\
\hline Category & Predefined budget options \\
\hline FURPS+ & Model for classifying software quality attributes\\
\hline Domain Model & Depiction of relationship between entities (person, concept, object or event) \\
\hline User Case Diagram & Depiction of association between actor and use cases \\
\hline
\end{tabular}
\caption{Data Dictionary}
\end{center}
\end{table}

\section{References}
Leave Debt Behind. (2010). 10 Recommended Category Percentages for Your Family Budget. [online] Available at: http://www.leavedebtbehind.com/frugal-living/budgeting/10-recommended-category-percentages-for-your-family-budget/ [Accessed 9 Feb. 2018].

\appendix


\end{document}
